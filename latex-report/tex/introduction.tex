\section{Введение}

Когда в начале 2019 года меня пригласили быть руководителем сборной РК по химии, я с удивлением обнаружил, что несмотря на то, что республиканская олимпиада школьников по химии (и связанные с ней мероприятия) проходит уже почти 30 лет, у руководителей сборной нет никаких данных, на которые можно было бы опираться при принятии решений. Мы не могли ответить на элементарные вопросы: насколько хорошо ученики справлялись с похожими заданиями в предыдущие годы? Насколько хорошо они пишут районную олимпиаду? А областную? Насколько хорошо областная олимпиада готовит учеников к заключительному этапу? А кто участвует на каждом этапе? Какое распределение по школам? Как много учеников неспециализированных школ? Насколько осведомлены участники разных этапов о доступных для них ресурсах?

Хочется верить, что за почти 5 лет вовлеченности в олимпийское движение, я смог хоть немного, но уменьшить долю хаоса и увеличить количество структурированной информации. Текущий документ - попытка свести собранную информацию об участниках РО в единый документ, в надежде, что последующие руководители сборных смогут пользоваться им для принятия \textit{data-driven decisions}. 

Код, с помощью которого был произведен анализ данных, доступен на \href{https://github.com/anmorgunov/respa-data-analysis}{Github}. Там же есть все графики, которые есть в этом документе. Если вы предпочитаете самостоятельно изучать информацию - приглашаю вас \href{https://github.com/anmorgunov/respa-data-analysis/tree/main/export/pdf}{в папку /export/pdf/}.


\textit{\\
С уважением,\\
Моргунов Антон\\
Председатель Коллегии и руководитель сборной (2019-2023)}