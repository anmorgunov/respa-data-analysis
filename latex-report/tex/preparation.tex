Поскольку решения, о которых пойдет речь в этой секции, в основном принимались председателем Коллегии (Моргунов А.), текст этого раздела написан от имени председателя Коллегии.

\section{Уроки предыдущего учебного года}

Концептуальные решения о том, какими будут задания районного, областного и заключительного этапа РО в этом году принимались под влиянием уроков и выводов, сделанных по результатам предыдущего цикла РО (см. \href{https://qazcho.kz/docs/reports/qazcho_respa_analysis.pdf}{Аналитическую записку по результатам РО-2022} и \href{https://qazcho.kz/docs/reports/2021-2022-report.pdf}{Отчет о работе Коллегии QazChO за 2021-2022 академический год}). 

\subsection{Сложность олимпиад}

В прошлом году было констатировано снижение уровня подготовки школьников (например, далко не все участники заключительного этапа справлялись с заданиями школьной программы). Поэтому, на ежегодном общем заседании Коллегии (\href{https://qazcho.kz/docs/public-protocols/protocol10.pdf}{Протокол №10 от 27 августа 2022}) было принято решение ввести разделение на «задачи с коротким условием» и «задачи с длинным условием» и принять некоторые правила о градации количества «задач, с коротким условием» по мере продвижения по этапам РО и классам участия. Осознанный фокус на создание определенного количества задач с коротким условием позволил значительно упростить сложность районного и областного этапов РО, изменение, которое было положительно оценено подавляющим большинством участников (см. Раздел 3. Восприятие олимпиады). На заключительном этапе в 10-11 классах была всего одна задача с коротким условием, что, возможно, отразилось на сложности комплекта - ученики посчитали, что заключительный этап был сложнее, чем в предыдущие годы. 

В дополнении, в этом учебном году, впервые в истории, все задания всех этапов РО оценивались на объем и сложность бывшими олимпиадниками, которые решали эти задания на время. Внедрение апробации позволило значительно сократить количество опечаток и ошибок в условиях, но, к сожалению, не привело к их полному исчезновению. В этом году задания решал только один олимпиадник (Касымалы Мадияр, золотой медалист IChO 2022); вполне возможно, что к апробации нужно привлекать больше одного человека. Примечательно, что несмотря на то, что Касымалы М. считал, что комплект заданий заключительного этапа сложнее, чем в предыдущие годы, он был крайне удивлен результатам учащихся: они были ниже, чем он ожидал. Тоже самое можно сказать и про остальных членов Коллегии: мы понимали, что задания немного сложнее, но не ожидали таких низких результатов (см. комментарии по задачам в разделе Результаты РО.) Вероятно, что к апробации стоит привлекать олимпиадников с варьирующимся уровнем подготовки, дабы лучше оценивать сложность олимпиады.

\subsection{Устойчивость процесса составления}

Предыдущий цикл РО можно назвать «травмирующим». К некоторым этапам, некоторые составители готовили по 3-5 задач. С одной стороны, небольшое количество составителей приводило к более сбалансированным и органичным комплектам заданий: во-первых, один и тот же составитель не будет делать две похожие задачи, во-вторых, коммуникация с меньшим количеством участников происходит быстрее и эффективнее. К сожалению, такой подход не является устойчивым в долгосрочной перспективе. После заключительного этапа РО-2022, я боялся, что мы можем потерять чуть ли не половину составителей, которые не захотят продолжать делать задания в последующие годы. Ситуация усугублялась хамскими, враждебными и пренебрежительными действиями РНПЦ Дарын (см. \href{https://qazcho.kz/docs/reports/qazcho_respa_analysis.pdf}{Аналитическую записку по результатам РО-2022}), которые приводили к полной потере желания что-либо делать. Учитывая такую обстановку, в этом году я принял решение постараться максимально снизить количество задач, которое составляет один автор, а также максимизировать свободу в действиях минимизируя количество требований и правок (я действовал из допущения, что учитывая то, что работа по составлению задач не является высокооплачиваемой, авторы могут получать моральное удовлетворение за счет возможности делать то, что хотят в первую очередь они сами). В результате, в Коллегию были привлечены 3 новых составителя, а составлением заданий РО занимались рекордные 14 человек (над одним заключительным этапом трудились, тоже рекордные, 12 человек).

\textbf{Составители заданий в цикле РО 2022-2023:}\vspace{-1em}
\begin{align*}
    &\text{Аманжолов Азим} & &\text{Жаксылыков Азамат} & &\text{Курамшин Болат} & &\text{Молдагулов Галымжан} \\
    &\text{Бегдаир Санжар} & &\text{Загрибельный Богдан} & &\text{Мадиева Малена} & &\text{Мужубаев Абильмансур} \\
    &\text{Бекхожин Жанибек} & &\text{Касьянов Артем} & &\text{Мельниченко Даниил} & &\text{Тайшыбай Айдын} \\
    &\text{Галикберова Милана} & &\text{Моргунов Антон} & &\text{} & &\text{} 
\end{align*}

В результате, один составитель предлагал не больше 2-3 задач, что значительно повысило устойчивость процесса подготовки комплектов заданий. Следующая задача - прийти к тому, чтобы количество предлагаемых задач на каждый этап превышало количество требуемых задач, дабы была возможность выбирать и создавать более органичный комплект. Для выполнения этой задачи необходимо и дальше привлекать новых составителей в Коллегию.

\subsection{Внедрение \LaTeX}

Традиционно, после получения всех задач, председатель Коллегии тратил 2-3 рабочих дня на компоновку заданий в единый комплект, что подразумевает приведение к общему формату, исправление маленьких ошибок, правильная нумерация, правильные и одинаковые отступы и так далее. Для минимизации времени, которое уходит на визуальное оформление комплекта заданий было принято решение перейти на оформление заданий в \LaTeX. Новый член Коллегии Жаксылыков А. разработал невероятно красивый и удобный шаблон (\href{https://github.com/Beyond-Curriculum/qazcho-latex}{доступный на Github}), который позволил свести на нуль время, необходимое на оформление заданий.


\section{Задания районного этапа}
Подготовка заданий началась 5 сентября 2022 г. Составителям был поставлен дедлайн 23 октября. Одновременно поставлены дедлайны для заданий областного (27 ноября) и заключительного (26 февраля) этапов. Было предложено следующее распределение по кол-ву задач с коротким и длинным условием:

\begin{table}[h]
    \centering
    \begin{tabular}{|c|c|c|}
        \hline
        Класс & Кол-во задач с коротким условием & Кол-во задач с длинным условием \\ \hline
        9 & 3 & 1 \\ \hline
        10 & 3 & 2 \\ \hline
        11 & 3 & 2 \\ \hline
    \end{tabular}
    \label{tab:tasks}
\end{table}

По темам составителям была предоставлена полная свобода. Единственное, что было принято решение не составлять задания по органической химии в 9 и 10 класс. В рамках \href{https://qazcho.kz/docs/npa/respa-syllabus.pdf}{Концепции тем и сложности РО (далее - силлабус)}, систематические знания органической химии ожидаются с областного этапа 10 кл. На всех этапах 9 кл. и районном этапе 10 кл. подразумевалось знание \textit{школьной органики}. Изначально, это понятие попало в силлабус поскольку в 9 кл. (особенно на заключительном этапе) традиционно были задачи по органике, но требовать системного понимания органики (например того, которое образуется после прочтения учебника \textit{John McMurry. Organic Chemistry}) раньше областного этапа 10 кл. было бы неразумным. На основании опыта предыдущих лет, к началу текущего цикла РО сформировалось четкое понимание, что задачи по \textit{школьной органике} крайне ограничены в своем разнообразии и в какой-то степени сильно типичны. Иными словами, либо каждый год Коллегия будет составлять примерно одни и те же задачи, либо так или иначе придется выходить за рамки \textit{школьной органики}. Это понимание подкреплялось наблюдением за некоторыми учениками 9 класса на \href{https://ask.bc-pf.org}{форуме Спроси}, которые начинали изучение учебников, предназначенных для учеников 10-11 классов.

\section{Задания областного этапа}
Подготовка заданий началась 18 ноября 2022 г. Было предложено следующее распределение по кол-ву задач с коротким и длинным условием:

\begin{table}[h]
    \centering
    \begin{tabular}{|c|c|c|}
        \hline
        Класс & Кол-во задач с коротким условием & Кол-во задач с длинным условием \\ \hline
        9 & 3 & 2 \\ \hline
        10 & 3 & 3 \\ \hline
        11 & 2 & 4 \\ \hline
    \end{tabular}
    \label{tab:tasks11}
\end{table}

Руководствуясь той же аргументацией, что и при подготовке заданий районного этапа, было решено не включать задания по органическому синтезу в 9 класс. Использование органических молекул и реакций в задачах на смеси и неизвестные вещества приветствовалось. 

При подготовке заданий областного этапа началось обсуждение концепции заключительного этапа, дабы задания области начинали готовить участников к заключительному этапу. Было предложено в этом году сделать фокус на т.н. научной грамотности (\textit{scientific literacy}): умение читать (интерпретировать)  графики, схемы, экспериментальные установки и т. д. В идеале, сделать что-то похожее на третью задачу IChO 2022 (основная сложность задачи сводилась к правильной интерпретации экспериментальной установки, сами расчеты были простыми), ибо такие задачи вызывают сложности у сборной РК. В результате, в комплекте заданий появился \textit{Органический блиц} (\href{https://olympiads.bc-pf.org/chemistry/oblast/2023/10}{Задача №10-5}) и \textit{Электрохимия возвращается} (\href{https://olympiads.bc-pf.org/chemistry/oblast/2023/11}{Задача №11-5}). 

\section{Задания заключительного этапа}

Подготовка заданий началась 23 января 2023 г. Было предложено следующее распределение по кол-ву задач с коротким и длинным условием:

\begin{table}[h]
    \centering
    \begin{tabular}{|c|c|c|}
        \hline
        Класс & Кол-во задач с коротким условием & Кол-во задач с длинным условием \\ \hline
        9 & 4 & 3 \\ \hline
        10 & 3 & 4 \\ \hline
        11 & 3 & 4 \\ \hline
    \end{tabular}
\end{table}

Руководствуясь той же аргументацией, что и на предыдущих этапах, было принято решение не делать задачу на органический синтез в 9 кл. Отдельно была сделана просьба делать 1-2 простых пункта/вопроса в каждой задаче.

Среди заданий, которые были предложены составителями к концу февраля, было недостаточно заданий с коротким условием. Более того, ряд заданий значительно отошли от изначальной задумки. К сожалению, на тот момент уже не оставалось времени (одна неделя выделялась на оформление, вычитку заданий и исправление ошибок, одна неделя на перевод) на значительные правки или составление новых задач, поэтому комплект был составлен из того, что было предложено. Скорее всего, в последующие годы рационально будет ставить дедлайн составления заданий заключительного этапа за два месяца до олимпиады, дабы было время заменить/значительно видоизменить предложенные задачи.