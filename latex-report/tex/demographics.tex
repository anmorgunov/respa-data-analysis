\section{Введение}

В 2022-2023 учебном году, комплекты заданий и решений теоретических туров районного, областного и заключительного этапов республиканской олимпиады школьников по химии сопровождались призывом оставить обратную связь о прошедшей олимпиаде, заполнив онлайн опрос, созданный Коллегией. Ссылка на опрос также дублировалась на официальных каналах Коллегии и на (\href{https://ask.bc-pf.org}{форуме Спроси}). В распространении опроса РНПЦ Дарын и Министерство Просвещения РК не принимали никакого участия. В рамках опроса, помимо «существенных» вопросов, ответы на которые проанализированы в других разделах, участники предоставляли демографические данные. Некоторые сводки настолько нас удивили, что мы решили посвятить им отдельную секцию.

Опрос по районному этапу РО прошли 40 участников. Из них 10 (25\%), 15 (37.5\%) и 15 (37.5\%) человек участвовали за 9, 10 и 11 класс соответственно. Распределение по школам следующее:

\begin{table}[ht]
    \centering
    \begin{tabular}{|l|c|c|}
    \hline
    \textbf{Тип школы} & \textbf{Количество} & \textbf{Процент} \\ \hline
    Обычная школа (в городе) & 16 & 40.0\% \\ \hline
    Обычная школа (в селе) & 12 & 30.0\% \\ \hline
    БИЛ & 8 & 20.0\% \\ \hline
    Другая специализированная школа & 2 & 5.0\% \\ \hline
    Частная школа & 2 & 5.0\% \\ \hline
    НИШ & 0 & 0.0\% \\ \hline
    РФМШ & 0 & 0.0\% \\ \hline
    \end{tabular}
    \caption{Типы школ и их распределение на районном этапе РО}
    \label{tab:region-schols}
\end{table}

Стоит заметить, что ученики НИШ и БИЛ не пишут районный этап и проходят сразу на областной, поэтому ответов от учеников НИШ нет. 8 ответов от учеников БИЛ скорее всего предоставлены учениками, которые самостоятельно прорешивали задания районного этапа.

Опрос по областному этапу РО прошли 60 участников. Из них 28 (46.7\%), 17 (28.3\%), и 15 (25.0\%) человек участвовали за 9, 10 и 11 класс соответственно. Распределение по школам следующее:

\begin{table}[ht]
    \centering
    \begin{tabular}{|l|c|c|}
        \hline
        \textbf{Тип школы} & \textbf{Количество} & \textbf{Процент} \\ \hline
        БИЛ & 31 & 51.7\% \\ \hline
        Другая специализированная школа & 11 & 18.3\% \\ \hline
        НИШ & 8 & 13.3\% \\ \hline
        Обычная школа (в городе) & 6 & 10.0\% \\ \hline
        Частная школа & 3 & 5.0\% \\ \hline
        РФМШ & 1 & 1.7\% \\ \hline
        Обычная школа (в селе) & 0 & 0.0\% \\ \hline
    \end{tabular}
    \caption{Типы школ и их распределение на областном этапе РО}
    \label{tab:obl-schools}
\end{table}

Нельзя не заметить практически полное отсутствие ответов от учников из неспециализированных школ. Конечно, 40 и 60 респондентов для этапа, на котором участвуют десятки тысяч (район) и тысячи (область) школьников, могут быть не репрезентативной выборкой. Однако, сложно представить, что ученики из неспециализированных школ, которые участвовали на районном этапе и заполняли опрос, не заполнили бы опрос на областном этапе, если бы они на нем участвовали. Таким образом, Таблица \ref{tab:obl-schools} может указывать на то, что ученики неспециализированных школ в своем большинстве не проходят дальше районного этапа.

Опрос по заключительному этапу РО прошли 32 участника. Из них 9 (28.1\%), 9 (28.1\%) и 14 (43.8\%) человек участвовали за 9, 10 и 11 класс соответственно. Распределение по школам следующее:

\begin{table}[ht]
    \centering
    \begin{tabular}{|l|c|c|}
    \hline
    \textbf{Тип школы} & \textbf{Количество} & \textbf{Процент} \\ \hline
    БИЛ & 15 & 46.9\% \\ \hline
    Другая специализированная школа & 5 & 15.6\% \\ \hline
    Обычная школа (в городе) & 5 & 15.6\% \\ \hline
    НИШ & 4 & 12.5\% \\ \hline
    Частная школа & 2 & 6.3\% \\ \hline
    РФМШ & 1 & 3.1\% \\ \hline
    Обычная школа (в селе) & 0 & 0.0\% \\ \hline
    \end{tabular}
    \caption{Типы школ и их распределение на заключительном этапе РО}
    \label{tab:respa-schools}
\end{table}

\section{Язык обучения и язык участия на РО}
